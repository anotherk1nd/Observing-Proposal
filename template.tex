\documentclass[11pt]{article}
\usepackage{template}
\usepackage{graphicx}
\usepackage{amssymb}
\usepackage{epstopdf}
\DeclareGraphicsRule{.tif}{png}{.png}{`convert #1 `dirname #1`/`basename #1 .tif`.png}

\begin{document}

%--------------------------------------------------------------------------
% Title of your proposal
%--------------------------------------------------------------------------
\proptitle{Proposal Title}

%--------------------------------------------------------------------------
% Abstract (replace/add content)
%--------------------------------------------------------------------------
\abstract{max. 1500 characters. }

%--------------------------------------------------------------------------
% Time/Instrument summary (modify fields as required)
%--------------------------------------------------------------------------
\totaltime{0\,hr}

%--------------------------------------------------------------------------
% Summary of time requested for each SFC3 filter
%--------------------------------------------------------------------------
\filterbreakdown{
{\it e.g.} & & &\\
\hline
$J_{f125w}$ & 2 & 10 minutes & 24 minutes \\
\hline
$H_{f160w}$ & 1 & 20 minutes & 22 minutes \\
}

%--------------------------------------------------------------------------
% Display intro section (do not remove or edit this line)
%--------------------------------------------------------------------------
\propintro

%--------------------------------------------------------------------------
% Science justification (about 1 page)
%--------------------------------------------------------------------------

\clearpage

\section{Science Motivation}

Give some background on the topic in general and describe the main aims of your proposed observations. This is limited to one page and should include appropriate references.\\

\noindent In doing so try to provide answers to the following:
\begin{itemize}
\item Which astrophysical questions will you be able to address with the requested data and how does this advance the field?
\item Why is the sample/object you plan to observe particularly suitable for addressing these questions?
\item Why are observations with the Hubble Space Telescope especially useful for addressing the key science questions you have identified?
\end{itemize}

\vfill \noindent {\bf References:} A. Fox et al. 2005, {\it Nature}, 444, 78; S. Nova \& B. L. Hole 1987, {\it The Astrophysical Journal}, 45, 8; T. Fringe et al. 2014, {\it Journal of Pure and Applied Nonsense}, 501, 999

%--------------------------------------------------------------------------
% Technical justification (about 1/2 page)
%--------------------------------------------------------------------------

\clearpage

\section{Technical Justification}

Motivate and describe your choice of instrument, observing strategy and observing time request. Don�t forget to add telescope overheads to the actual observing time. Explain what the expected brightness/flux of your sources is and what the significance level is, with which you are aiming to detect your targets (e.g. signal-to-noise ratio S/N=10). This can be as long as you need.


\end{document}  